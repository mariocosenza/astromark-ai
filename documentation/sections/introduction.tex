%! Author = giuliosacrestano
%! Date = 03/02/25

\chapter{Introduzione}

\sloppy
Astromark AI è un'applicazione di machine learning basata su Python, progettata per la classificazione avanzata di testi. Costruita con Flask, espone i suoi servizi tramite endpoint RESTful, facilitando l'integrazione con diversi sistemi. L'applicazione sfrutta moderne tecniche di apprendimento automatico per fornire un'analisi testuale ad alte prestazioni, adatta a compiti come l'analisi del sentiment, la categorizzazione dei contenuti e la classificazione tematica.

\section{Caratteristiche principali}

\begin{itemize}
    \item \textbf{Potenza del Machine Learning}: utilizza algoritmi e modelli avanzati per una classificazione testuale accurata ed efficiente.
    \item \textbf{Integrazione Restful}: offre servizi di classificazione testuale attraverso un'API REST, semplificando l'integrazione in diverse applicazioni.
    \item \textbf{Tecnologia Python e Flask}: sviluppata con Python e Flask, garantisce un'architettura modulare, leggera ed estensibile.
    \item \textbf{Documentazione completa}: include una documentazione dettagliata dell'API per facilitare l'onboarding e l'utilizzo.
\end{itemize}

Astromark AI è la soluzione ideale per sviluppatori e ricercatori alla ricerca di una piattaforma affidabile e di facile utilizzo per la classificazione testuale.
\section{Specifica PEAS}
Di seguito è riportata la descrizione PEAS dell'ambiente operativo di Astromark-AI.
\begin{table}[h]
\centering
\begin{tabular}{|p{4cm}|p{10cm}|}
\hline
\multicolumn{2}{|c|}{\textbf{PEAS}} \\
\hline
\textbf{Performance} & Criteri per valutare il successo dell'agente, ad esempio la percentuale di risposte corrette e utili , il tempo medio di risposta. \\
\hline
\textbf{Environment} & L'ambiente in cui l'agente opera: i ticket inviati, da genitori o docenti, alla Piattaforma Astromark per riceve assistenza su una problematica. \\
\hline
\textbf{Actuators} & I mezzi con cui l'agente agisce sull'ambiente, Astromarl-AI comunicare con l'ambiente organizzando i ticket in base alla categoria. \\
\hline
\textbf{Sensors} & I canali attraverso cui l'agente percepisce l'ambiente, input testuali , nello specifico i ticket . \\
\hline
\end{tabular}
\caption{Specifica PEAS}
\label{tab:peas}
\end{table}

\FloatBarrier

\section{Caratteristiche ambiente}
L'ambiente operativo è:
\begin{itemize}
    \item \textbf{Completamente Osservabile: }possiamo accedere ai ticket in qualsiasi momento
    \item \textbf{Deterministico: }lo stato successivo dell’ambiente è completamente determinato dallo stato corrente e dall’azione eseguita dall’agente
    \item \textbf{Sequenziale: }le scelta dell'azione non dipende dal singolo episodio
    \item \textbf{Statico: }l'ambiente rimane invariato mentre l'agente sta deliberando.
    \item \textbf{Discreto: }l'ambiente fornisce un numero limitato di percezioni
    \item \textbf{Singolo: }l'ambiente consente la presenza di un unico agente.
\end{itemize}

