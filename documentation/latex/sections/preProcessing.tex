\chapter{Pre Processing}

\section{Introduzione}
La fase di pre-processing approfondisce la trasformazione del testo già estratto e pulito, arricchendo la rappresentazione semantica mediante operazioni quali tokenizzazione, lemmatizzazione e Named Entity Recognition (NER). Queste tecniche avanzate migliorano l'analisi computazionale e preparano il testo per algoritmi di machine learning.

\section{Tokenizzazione, Lemmatizzazione e Named Entity Recognition}
Per incrementare il valore semantico del testo, vengono applicate le seguenti operazioni:
\begin{itemize}
    \item \textbf{Tokenizzazione:} segmenta il testo in unità minime (token).
    \item \textbf{Lemmatizzazione:} riduce ogni token alla sua forma base, diminuendo la variabilità morfologica.
    \item \textbf{Named Entity Recognition (NER):} identifica entità come nomi, organizzazioni e località, etichettandole come ad esempio \texttt{NER\_PERSON}.
\end{itemize}

La seguente funzione integra questi passaggi:

\begin{lstlisting}[language=Python,caption={Funzione per il processing del testo}]
def process_text(text):
    cleaned_text = minimal_preprocess(text)
    doc = nlp(cleaned_text)
    tokens = []
    for token in doc:
        if token.is_stop or token.is_punct or token.is_space:
            continue
        lemma = token.lemma_.strip()
        if lemma:
            tokens.append(lemma)
    for ent in doc.ents:
        tokens.append("NER_%s" % ent.label_)
    return ' '.join(tokens)
\end{lstlisting}

\subsection{Deep Learning in spaCy per il Riconoscimento delle Entità} spaCy sfrutta modelli di deep learning che combinano:
\begin{itemize}
\item \textbf{Reti Neurali Convoluzionali (CNN):} per estrarre informazioni locali dal testo.
\item \textbf{Meccanismi di Attenzione:} per comprendere il contesto attorno a ciascun token.
\item \textbf{Architetture Trasformative:} per individuare relazioni tra token anche se distanti nel testo.
\end{itemize} Queste tecnologie consentono di raggiungere un elevato livello di accuratezza nel riconoscimento delle entità, anche in testi complessi.

\section{Parallelizzazione del Pre-processing}
Per gestire dataset di grandi dimensioni, la parallelizzazione sfrutta la libreria \texttt{joblib} per distribuire il processo su più core. Il backend "threading" viene utilizzato per ottimizzare l'uso delle risorse hardware. Esempio:

\begin{lstlisting}[language=Python,caption={Funzione parallel\_process\_texts}]
def parallel_process_texts(series, n_jobs=-1):
    logger.info("Parallel text processing with threading backend...")
    with parallel_backend('threading', n_jobs=n_jobs):
        processed = Parallel()(delayed(process_text)(text) for text in series)
    return pd.Series(processed, index=series.index)
\end{lstlisting}